\documentclass[twocolumn]{article}

\usepackage{amsmath, amssymb, amsthm}
\usepackage{tikz}
\usetikzlibrary{arrows,decorations.markings}
\usetikzlibrary{calc}
\usetikzlibrary{intersections}
\usetikzlibrary{positioning}
\usetikzlibrary{patterns}
\usetikzlibrary{circuits.logic.US,circuits.logic.IEC,fit}
\newcommand\addvmargin[1]{
  \node[fit=(current bounding box),inner ysep=#1,inner xsep=0]{};
}
\newcommand\yt{http://www.youtube.com/embed/IwLSrNu1ppI?autoplay=1&controls=0&loop=1&start=01}
\usepackage{pgfplots}
\usepgflibrary{patterns}
\usepackage[T1]{fontenc}
\usepackage{mathptmx}
\usepackage{multirow}
\usepackage{textcomp}
\usepackage{multicol}
\usepackage{coffee}
\usepackage{makecell}
\usepackage[makeroom]{cancel}
\usepackage{lscape}
\usepackage[utf8]{inputenc}
\usepackage{geometry}
 \geometry{
	 paperwidth=147cm,
	 paperheight=147cm,
	 left=30mm,
	 right=10mm,
	 top=30mm,
	 bottom=20mm,
	 }
\usepackage{hyperref}
\newcommand{\thicc}{thick}
\newcommand{\ultraskip}{\bigskip\bigskip\bigskip\bigskip\bigskip\bigskip}
\newcommand{\tabellenskalierung}{0.75}
\newcommand{\PreserveBackslash}[1]{\let\temp=\\#1\let\\=\temp}
\newcolumntype{C}[1]{>{\PreserveBackslash\centering}p{#1}}
\newcolumntype{R}[1]{>{\PreserveBackslash\raggedleft}p{#1}}
\newcolumntype{L}[1]{>{\PreserveBackslash\raggedright}p{#1}}

\newcommand*{\info}[4][16.3]{%
	\node [ annotation, #3, scale=0.65, text width = #1em, inner sep = 2mm ] at (#2) {%
	\list{$\bullet$}{\topsep=0pt\itemsep=0pt\parsep=0pt
	\parskip=0pt\labelwidth=8pt\leftmargin=8pt
	\itemindent=0pt\labelsep=2pt}%
	#4
	\endlist
	};
}

\usepackage{CrimsonPro}
\usetikzlibrary{mindmap,trees}
\begin{document}
\pagestyle{empty}

\begin{tikzpicture}[	
			mindmap, 
			concept, 
			every node/.style = {concept,minimum size=0cm}, 
			concept color=darkgray,
			text=white,
			grow cyclic,
			text width=3.5cm, 
			%align=flush center,
			level 1/.style={level distance=10cm, sibling angle=35, font=\large},
			level 2/.style={level distance=7.5cm, sibling angle=35, font=\large},
			level 3/.style={level distance=7cm, sibling angle=43, font=\large}, 
			level 4/.style={level distance=6cm, sibling angle=40,  font=\normalsize}
		]
	\draw (10,20) node[text width=70mm] {\href{\yt}{\Huge \textbf{Staat}}}
[clockwise from =100]
child { node (gg) {Grundgesetz}
[clockwise from=170]
	child {node {Funktionen}
	[clockwise from=270]
		child {node {Abwehrfunktion}}
		child {node {Teilhabefunktion}}
		child {node {Leistungsfunktion}}
		child {node {Werteordnung }}
		}
	child {node {Schutz}
	[clockwise from=130]
		child { node {Schutz des Individuums}}
		child { node {Schutz der Kommunikations mit anderen}}
		child { node {Schutz der Erwerbstätigkeit und ihrer Ergebnisse }}
		}
	}
child  { node {Bestandteile} 
	child { node { Staatsgebiet}}
	child { node { Staatsvolk}}
	child { node { Staatsgewalt}} }
child[yshift=7cm, xshift=7cm] { node {Staatsstruktur\-\-prinzip}
[clockwise from=110]
	child[sibling angle=30] { node {Demokratie\-prinzip}}
	child[sibling angle=30] { node {Sozialstaats\-prinzip}}
	child[sibling angle=30] { node {Rechtstaats\-prinzip}}
	child[sibling angle=30] { node {Gewaltenteilungs\-prinzip}}
	child[sibling angle=30] { node {Bundesstaat\-prinzip}}
	child[sibling angle=30] { node {Volkssouveränität}}
	}
child { node {Bundesorgane}
[clockwise from=70]
	child[sibling angle=30] { node {Bundestag}}
	child[sibling angle=30] { node {Bundesrat}}
	child[sibling angle=30] { node {Bundesregierung}}
	child[sibling angle=30] { node {Bundespräsident}}
	child[sibling angle=30] { node {Bundesverfassungs\-gericht}}
	}

	;
\draw (45,32) node[text width=65mm] {\Huge{\textbf{Privat\-autonomie} \Large Recht private Rechtsverträge nach eigener Entscheidung zu gestalten Art. 2 GG}}
	[clockwise from=120]
	child { node {Vertragsfreiheit}
	[clockwise from=150]
		child { node {Abschlussfreiheit}}
		child { node {Inhaltsfreiheit}}
		child { node {grundsätzliche Formfreiheit}}
		child { node {\textbf{Ausnahme} Kontrahierungszwang}}
		}
	child { node {Testierfreiheit}}
	child { node {Eigentumsfreiheit}}
	child { node {Grenzen}
	[clockwise from=65]
		child { node {Schutz der Schwächeren Vertragspartei $\rightarrow$ gerechtes Ergebnis}}
		child { node {Verbote}
			child { node {Gesetzliches Verbot}}
			child { node {Veräußerungs\-verbot}}
			child { node {Verstoß gegen die gute Sitte}}
			}
			}
 ;
\draw (-30,25) node[text width=70mm] {\Huge \textbf{Gesetz\-gebung}}
child { node {\textbf{Verwaltungsakt:} hoheitliche Maßnahme einer Behörde die Konkrete Einzelfälle regelt und Rechtsfolgen herbeiführen kann}}
child { node {\textbf{Gesetzgeber}}
	child {node {Gemeinden}}
	child {node {Länder}}
	child {node {Bund}} }
child { node {\textbf{ausschließlich}}}
child { node {\textbf{konkurrierend:} Recht Gesetze zu geben, wenn der andere von Seiner Befugnis keinen Gebrauch macht}};

\draw (50,-5) node[text width=70mm] {\href{\yt}{\Huge \textbf{Recht}}}
[clockwise from = 152]
child { node {\textbf{Subjektive:} Ansprüche des Einzelnen, die aus den Rechtsnormen hervorgehen}
[clockwise from = 140]
	child { node { \textbf{absolut:} gegenüber allen (eg. Eigentum)}}
	child { node { \textbf{relative:} gegenüber einzelnen (eg. Vertrag)}} }
child[level distance=20cm] { node {\textbf{Geschrieben/ Ungeschrieben} Gesetzestexte/ Gewohnheitsrecht}
[clockwise from=180]
	child { node {Verfassungs\-normen}}
	child { node {Gesetzliche Normen}}
	child { node {Verordnungen}}
	child { node {Satzungen}}}
child[yshift=2cm] { node {\textbf{Funktionen:}}
	 child { node {Ordnungsfunktion}}
	 child { node {Gerechtigkeits\-funktion}}
	 child { node {Herrschafts- und Kontrollfunktion}}
	 child { node {Friedenssicherung}}
	 child { node {Freiheitssicherung}} } 
child { node {\textbf{Zwingend:} Darf vertraglich nicht abgeändert werden \\ \textbf{Nachgiebig:} steht zur Disposition }} 
child[yshift=2cm,xshift=2cm] { node {\textbf{Vorrang:} bei der Anwendbarkeit von mehreren Rechtsnormen auf einen Sachverhalt}
[clockwise from = 70]
	child { node {lex-specialis}}
	child { node {lex-posterior}}
	child { node {lex-superior}
	[clockwise from=30, text width=50mm]
		child { node {\begin{enumerate}
				\item Europarecht
				\item Grundgesetz
				\item Formelle Gesetze \\ (z.B BGB)
				\item Materielle Gesetze
		\end{enumerate}}}
	}}
child { node {\textbf{Abgrenzung}}
[clockwise from=30]
	child { node {\textbf{Bräuche} Konformitätserwartung, kaum Repressalien bei Missachtung}}
	child { node {\textbf{Sitten} Stärkere Konformitätserwartung, negative Wahrnehmung in der Gesellschaft}}
	child { node {\textbf{Recht} Konformitätszwang, Bestrafung durch den Staat bei nichtbeachtung}}}
child { node {\textbf{Formell:} Verfahrensrecht (eg. §136 StPo)}} 
child { node {\textbf{Materiell:}  für die Rechtslage entscheidend }} 
child { node {\textbf{Objektive:} Summe aller rechtlichen Normen}
[clockwise from = 270]
	child { node {\textbf{Öffentlich:} Recht zwischen Staat und Bürgern, gekennzeichnet durch eine (Unterordnung)}
		child { node {Staats und Verwaltungsrecht}}
		child { node {Strafrecht}}}
	child { node {\textbf{Zivil:} Recht zwischen Bürgern und Bürgern, (Gleichordnung)}} }
child { node {\textbf{Eigenschaften}}
[clockwise from = 240]
	child { node {allgemein}}
	child { node {abstrakt}}
	child { node {verbindlich}}
	child { node {bekannt zu machen}}};

 \draw (0,-30) node[text width=70mm] {\href{\yt}{\Huge \textbf{Rechtsgeschäfte}}}
 		child { node {zweiseitige Rechtsgeschäfte}
			child { node {\textbf{Vertrag}:\\ besteht aus zwei übereinstimmenden Willenserklärungen}
				child { node {Angebot}}
				child { node {Annahme}}
				child { node {\textbf{\href{https://www.gesetze-im-internet.de/bgb/__433.html}{§ 433}} Vertragstypische Pflichten beim Kaufvertrag}}}}
		child[level distance=7.5cm, sibling angle=37] { node {einseitige Rechtsgeschäfte}
			child { node {Testament}}
			child { node {Kündigung}}
			child { node {Widerruf}} }
		child[level distance=14cm, sibling angle=40] { node {\textbf{Willenserklärung} \\ Willensäußerung die auf eine Rechtsfolge gerichtet ist.}
		[clockwise from=80]
			child { node {Invitaio ad offerendum \\ Sonderfall, das Angebot an die Allgemeinheit}}
			child[level distance=10.5cm] { node {Wirksamwerden}
					child { node {Empfangsbedürftige Willenserklärung}
						[clockwise from=120]
						child { node {\textbf{Abgabe} \\ auf den Weg bringen zum Empfänger}}
						child { node {\textbf{Zugang §130} \\ (nicht wirksam wenn gleichzeitig ein Widerruf zugeht)}}
						child[sibling angle=43] { node {\textbf{Unter anwesenden} \\ (Auch Telefonat o.Ä.) \\ Zugang, sofort, wenn der Erklärende davon ausgehen kann vollständig verstanden worden zu sein.}
						}}
					child { node {nicht Empfangsbedürftige Willenserklärung}} }
				child { node {Teile}
				[clockwise from=35]
				child { node {objektiver Tatbestand \\ die äußere Erklärung \\ ausdrücklich oder konkludent}}
				child { node {subjektiver Tatbestand \\ der Erklärungswille}
					child { node {Handlungswille}}
					child { node {Erklärungsbewusstsein}}
					child { node {Geschäftswille}
					[clockwise from=-30]
						child { node {Willenstheorie}}
						child { node {Erklärungstheorie}}}
					}
			}	
			child[sibling angle=40] { node {\textbf{Auslegung} \\ dem objektivem Empfängerhorizont, das heißt Nach dem Verständnis eines Verständigen aber außenstehendem Dritten.} 
			[clockwise from=340]
				child[level distance=7cm] { node {\textbf{\href{https://www.gesetze-im-internet.de/bgb/__133.html}{§ 133}} \\ Bei der Auslegung einer Willenserklärung ist der Wirkliche Wille zu erforschen und nicht and em buchstäblichen Sinne des Ausdrucks zu haften.}}
				child[sibling angle=50,level distance=7cm] { node {\textbf{\href{https://www.gesetze-im-internet.de/bgb/__157.html}{§ 157}} \\ Verträge sind so auszulegen, wie Treu und Glauben mit Rücksicht auf die Verkehrssitte es erfordern.  }}
				child[sibling angle=50,level distance=7cm] { node {\textbf{Falschbezeichnung} schadet nicht, wenn sie übereinstimmend geschieht.}}
			}
			}
		child { node {\textbf{Anspruch} \\ Recht, von einem anderen ein Tun oder Unterlassen zu fordern }}
	child[level distance=18cm, sibling angle=23] { node {\textbf{Geschäftsfähigkeit} \\ Fähigkeit, selbst durch eigen Willenserklärung Rechtsfolgen herbeizuführen.}
	[clockwise from=90]
		child[] { node {\textbf{\href{https://www.gesetze-im-internet.de/gg/art_2.html}{§ 2}} \\ Eintritt in die Volljährigkeit}}
		child[level distance=15cm] { node {Geschäftsunfähigkeit}
		[clockwise from=130]
			child { node {\textbf{\href{https://www.gesetze-im-internet.de/bgb/__105a.html}{§ 105a}}\\ Geschäfte des täglichen Lebens}}
			child { node {\textbf{\href{https://www.gesetze-im-internet.de/bgb/__104.html}{§ 104}} \\ Geschäfts\-unfähigkeit }}
			child { node {Rechtliche Betreuung \\ \textbf{\href{https://www.gesetze-im-internet.de/bgb/__1896.html}{§ 1896}} Voraussetzungen \\ Beschränkte Geschäftsfähigkeit Minderjähriger}}
			child { node {\textbf{\href{https://www.gesetze-im-internet.de/bgb/__1626.html}{§ 1626}} \\ Elterliche Sorge \\ \textbf{\href{https://www.gesetze-im-internet.de/bgb/__1629.html}{§ 1629}} \\ Vertretung des Kindes}}
			}
		child { node {beschränkte Geschäftsfähigkeit}
		[clockwise from=30]
			child { node {\textbf{\href{https://www.gesetze-im-internet.de/bgb/__110.html}{§ 110}} \\ Bewirken der Leistung mit eigenen Mitteln}}
			child { node {\textbf{\href{https://www.gesetze-im-internet.de/bgb/__108.html}{§ 108}} Vertragsschluss ohne Einwilligung \\ (Schwebende Unwirksamkeit)}}
			child { node {\textbf{\href{https://www.gesetze-im-internet.de/bgb/__107.html}{§§ 107}} \\ Einwilligung des gesetzlichen Vertreters}}
			child { node {\textbf{\href{https://www.gesetze-im-internet.de/bgb/__106 .html}{§§ 106 ff.}} \\ Beschränkte Geschäftsfähigkeit Minderjähriger}}
		}}
		child[sibling angle=40] { node {\textbf{Stellvertretung} \\ ist bei allen Rechtsgeschäften außer bei höchstpersönlich z.B Testament, Heirat}
			[clockwise from=90]
				child[yshift=2cm,xshift=-1cm] { node {\textbf{eigene Willens\-erklärung} \\ das heißt Entscheidungsspielraum}
				child { node {\textbf{\href{https://www.gesetze-im-internet.de/bgb/__165.html}{§ 165}} Beschränkt geschäftsfähiger Vertreter} }
				}
			child { node {handeln in \textbf{fremdem} \textbf{Namen}}
			[clockwise from=92]
				child[text width=50mm, level distance=8cm] { node {\textbf{\href{https://www.gesetze-im-internet.de/bgb/__164.html}{§ 164}} Wirkung der Erklärung des Vertreters \\ \textbf{Offenkundigkeitsprinzip} \\ es muss konkludent ersichtlich sein, wen der Vertreter vertritt}
				child[level distance=7cm] { node {Ausnahme: Geschäft für den, den es Angeht}}}
				}
			child { node {\textbf{Vertretungsmacht}}
			[clockwise from=90]
				child { node {\textbf{\href{https://www.gesetze-im-internet.de/bgb/__166.html}{§ 166}} Willensmängel; Wissenszurechnung}}
				child[level distance=7.3cm] { node {Vollmacht}
				[clockwise from=110]
					child[] { node {\textbf{\href{https://www.gesetze-im-internet.de/bgb/__171.html}{§ 171}} Wirkungsdauer bei Kundgebung \\ kundgemachte Innenvollmacht}}
				child { node {\textbf{\href{https://www.gesetze-im-internet.de/bgb/__167.html}{§ 167 II}} Erteilung der Vollmacht \\ Formfreiheit außer bei Risikobehafteten Geschäften}}
					child[sibling angle=40, level distance=7cm] { node {\textbf{\href{https://www.gesetze-im-internet.de/bgb/__167.html}{§ 167 I}} Erteilung der Vollmacht \\ Kann sowohl dem Vertreter gegenüber erklärt werden (\textbf{Innenvollmacht}) auch dem Dritten \textbf{Außenvollmacht}}}
				child[] { node {rechtliches Können im Außenverhältnis \\ rechtliches Dürfen im Innenverhältnis}
				[clockwise from=37]
					child { node {\textbf{\href{https://www.gesetze-im-internet.de/bgb/__280.html}{§ 280}} Schadensersatz wegen Pflichtverletzung }}
					child { node {\textbf{\href{https://www.gesetze-im-internet.de/bgb/__665.html}{§ 665}} Abweichung von Weisungen}} }
					}
				child[level distance=5.8cm] { node {Vertreter ohne Vertretungsmacht \\ \textbf{\href{https://www.gesetze-im-internet.de/bgb/__177.html}{§ 177}} \\ \textbf{\href{https://www.gesetze-im-internet.de/bgb/__178.html}{§ 178}} \\ \textbf{\href{https://www.gesetze-im-internet.de/bgb/__179.html}{§ 179}} }}
				}
				child[level distance=17cm, yshift=5cm] { node {Anschein und Duldungs\-vollmacht} 
				[clockwise from=10]
					child[level distance=9cm] { node {Duldungs\-vollmacht}
					[clockwise from=65]
						child { node {auftreten als Vertreter}}
						child { node {Vertretung wird wissentlich geduldet}}
						child { node {\textbf{\href{https://www.gesetze-im-internet.de/bgb/__242.html}{§ 242}} Leistung nach Treu und Glauben \\ Dritter nimmt Innenvollmacht an}}
						}
					child[sibling angle=40] { node {Anscheins\-vollmacht}
					[clockwise from=03]
						child[sibling angle=40] { node {wiederholtes auftreten als Vertreter}}
						child[sibling angle=40] { node {fahrlässige Unkenntnis des Vertretenden}}
						child[sibling angle=40] { node {\textbf{\href{https://www.gesetze-im-internet.de/bgb/__242.html}{§ 242}} Leistung nach Treu und Glauben \\ Dritter nimmt Innenvollmacht an}}
						}
						}
				child { node {Missbrauch der Vertretungsmacht}
				[clockwise from=29]
					child { node {Wider dem Innenverhältnis \\ Zu lasten des Vertretenden außer \textbf{\href{https://www.gesetze-im-internet.de/bgb/__138.html}{§ 138}} Sittenwidriges Rechtsgeschäft; Wucher}}
					child { node {Wider dem Außenverhältnis ?}}
				}
				}
child[sibling angle=70] { node {Widerrufsrecht}
	child { node {Verbraucher\-schützende Widerrufsrechte}
		child { node {Begriffe}
			child { node {\textbf{\href{https://www.gesetze-im-internet.de/bgb/__13.html}{§ 13}} Verbraucher}}
			child { node {\textbf{\href{https://www.gesetze-im-internet.de/bgb/__14.html}{§ 14}} Unternehmner}}}
		child[level distance=16cm] { node {\textbf{\href{https://www.gesetze-im-internet.de/bgb/__355.html}{§ 355}} Widerrufsrecht bei Verbraucherverträgen}
			child { node {\textbf{\href{https://www.gesetze-im-internet.de/bgb/__506.html}{§ 506}} Zahlungsaufschub, sonstige Finanzierungshilfe}}
			child { node {\textbf{\href{https://www.gesetze-im-internet.de/bgb/__495.html}{§ 495}} Verbraucherdarlehensverträge}}
			child { node {\textbf{\href{https://www.gesetze-im-internet.de/bgb/__485.html}{§ 485}} Widerruf bei Teilzeit-Wohnrechtev., V. über langfristige Urlaubsprodukte, Vermittlungsv. Tauschsystem}}
			child { node {\textbf{\href{https://www.gesetze-im-internet.de/bgb/__312g.html}{§ 312g}} Widerrufsrecht}}
			child { node {\textbf{\href{https://www.gesetze-im-internet.de/bgb/__312c.html}{§ 312c}} Fernabsatzverträge}}
			child { node {\textbf{\href{https://www.gesetze-im-internet.de/bgb/__312b.html}{§ 312b}} Außerhalb von Geschäftsräumen geschlossene Verträge}}
	}
	}
	child { node {Allgemeines Widerrufsrecht}
		child[text width=60mm] { node {\textbf{\href{https://www.gesetze-im-internet.de/bgb/__130.html}{§ 130}} Jeder kann seine Willenserklärung widerrufen, wenn der Widerruf vorher oder gleichzeitig mit ihr dem Empfänger zugeht.}}
	}
	}
 		;
 \draw (55,-55) node[text width=65mm] {\Huge \textbf{Schuldverhältnis} \Large Kraft des Schuldverhältnisses ist der Gläubiger berechtigt, von dem Schuldner eine Leistung zu fordern. Die Leistung kann auch in einem
 Unterlassen bestehen.\\ \textbf{\Huge \href{https://www.gesetze-im-internet.de/bgb/__241.html}{§ 241}} }
 child[text width=60mm] { node {Trennungs- und Abstraktionsprinzip \\Das Sachenrechtliche \textbf{Verfügungsgeschäft} muss vom Schuldrechtlichen \textbf{Verpflichtungsgeschäfte} getrennt werden}
 [clockwise from=00]
	child[level distance=10cm, text width=63mm] { node {Verpflichtungsgeschäft: \\ Jedes Rechtsgeschäft, das lediglich die Verpflichtung zu einer Leistung Begründet und damit ein Schuldverhältnis zum Entstehen bringt. z.B \textbf{\href{https://www.gesetze-im-internet.de/bgb/__433.html}{§ 433}} Kaufvertrag}}
	child[level distance=10cm, sibling angle=53, text width=63mm] { node {Verfügungsgeschäft jedes Rechtsgeschäft, das unmittelbar darauf gerichtet ist, die Rechtslage zu ändern (z.B. durch Veränderung, Übertragung, Belastung, Aufhebung eines Rechts) z.B \textbf{\href{https://www.gesetze-im-internet.de/bgb/__929.html}{§ 929 Übereignung}}}}
	}
child { node {Entstehung von Schuldverhältnissen}
	child { node {\textbf{\href{https://www.gesetze-im-internet.de/bgb/__311.html}{§ 311 I}} Rechtsgeschäft}
		child { node {Vertragsschluss}
			child { node {Gegenseitige Verträge \\ (synallagmatisch) \\ \textbf{\href{https://www.gesetze-im-internet.de/bgb/__323.html}{§ 323}} \textbf{326}}}
			child[level distance=8cm] { node {unvollkommen zweiseitige verpflichtende Verträge \\ \textbf{\href{https://www.gesetze-im-internet.de/bgb/__662.html}{§ 662}} Auftrag \\ \textbf{\href{https://www.gesetze-im-internet.de/bgb/__688.html}{§ 688}} Verwahrung \\ \textbf{\href{https://www.gesetze-im-internet.de/bgb/__598.html}{§ 598}} Leihe\\\textbf{\href{https://www.gesetze-im-internet.de/bgb/__516.html}{§ 516}} Schenkung}} 
			child { node {Einseitig Verplichtender Vertrag z.B \textbf{\href{https://www.gesetze-im-internet.de/bgb/__518.html}{§ 518}} Schenknung oder \textbf{§765} Bürgschaft}}}
		child[level distance=12cm] { node {Einseitige Rechstgeschäfte}
			child { node {\textbf{\href{https://www.gesetze-im-internet.de/bgb/__657.html}{§ 657}}Auslobung}}
			child { node {Vermächtnis}}
			child { node {Annahme von lieferung unbestellter Waren}}
			}
			}
		child[sibling angle=50] { node {Kraft Gesetzes}
		[clockwise from=100]
			child { node {\textbf{\href{https://www.gesetze-im-internet.de/bgb/__823.html}{§ 823}}   Unerlaubte Handlung}}
			child { node {\textbf{\href{https://www.gesetze-im-internet.de/bgb/__812.html}{§ 812}} ungerechtfertige Bereicherung}}
			child { node {\textbf{\href{https://www.gesetze-im-internet.de/bgb/__311.html}{§ 311 II}} \\Verschulden vor Vertragsschluss}
			[clockwise from=30]
				child { node {Vertragsverhandlungen}}
				child { node {Anbahnung eines Vertrages}}
				child { node {ähnliche Geschäftliche Kontakte}}
			}
			child { node {\textbf{\href{https://www.gesetze-im-internet.de/bgb/__677.html}{§ 677}} Geschäftsfürhung ohne Auftrag}}
			}
			}
child[yshift=4cm, level distance=20.5cm] { node {Pflichten aus dem Schuldverhältnis}
	[clockwise from=40]
	child { node {Die Schuld muss bestimt oder Bestimmbar sein}}
	child { node {\textbf{\href{https://www.gesetze-im-internet.de/bgb/__241.html}{§ 241 I}} Leistungspflicht}
		child { node {Hauptleistungs\-pflichten}}
		child { node {Nebenleistungs\-pflichten \\ Vertraglich vereinbart oder aus \textbf{\href{https://www.gesetze-im-internet.de/bgb/__242.html}{§ 242}} Treu und Glauben}}
		}
	child { node {\textbf{\href{https://www.gesetze-im-internet.de/bgb/__241.html}{§ 241 II}} Schutzpflicht}}
	}
child[xshift=-1cm,level distance=30cm] { node {\textbf{\href{https://www.gesetze-im-internet.de/bgb/__362.html}{§ 362}} Erlöschen durch Leistung}
	[clockwise from=94]
		child { node {Der Richtige \textbf{Schuldner} oder Dritter muss leisten, außer bei höchstpersönlicher Schuld}}
		child { node {Der Richtige \textbf{Gläubiger} kann Wechseln (z.B) Erbfall}}
		child { node {Die Richtige \textbf{Leistung} \\ \textbf{\href{https://www.gesetze-im-internet.de/bgb/__266.html}{§ 266}} der Schuldner ist nicht zu Teilleistungen berechtigt}
		[clockwise from=50]
			child { node {Stückschuld}}
			child { node {\textbf{\href{https://www.gesetze-im-internet.de/bgb/__243.html}{§ 243}} Gattungsschuld}}
			}
		child[text width=50mm, sibling angle=45] { node {Der Richtige \textbf{Leistungsort} \\ \textbf{\href{https://www.gesetze-im-internet.de/bgb/__269.html}{§ 269}} \\ Bestimmt wo der Leistungs\textbf{erfolg} eintritt. \\ Leistungsort $\neq$ Erfolgsort}
		[clockwise from=29]
			child[text width=30mm] { node {Holschuld \\ \textbf{Regelfall}}}
			child[text width=30mm] { node {Bringschuld}}
			child[text width=30mm] { node {\textbf{\href{https://www.gesetze-im-internet.de/bgb/__447.html}{§ 447}} Schickschuld \\ Käufer trägt das Transportrisiko}
			[clockwise from=0]
				child { node {\textbf{\href{https://www.gesetze-im-internet.de/bgb/__270.html}{§ 270}} Qualifizierte Schickschuld \\ Der Schuldner Trägt das Transportrisiko (z.B. Geldschuld)}}}
		}
		child[sibling angle=45, level distance=6cm] { node {Die Richtige \textbf{Leistungszeit} \\ \textbf{\href{https://www.gesetze-im-internet.de/bgb/__271.html}{§ 271}} Die Leistung ist sofort fällig und darf sofort Bewirkt werden.}}
}
child { node {Erfüllungssurrogat}
[clockwise from=43]
	child { node {Leistung an Erfüllungs Statt}}
	child { node {Hinterlegung}
		child { node {\textbf{\href{https://www.gesetze-im-internet.de/bgb/__372.html}{§ 372}} Wenn der Gläubiger im Annahmeverzug ist und die Sache hinterlegungs\-fähig ist.}}
		child { node {\textbf{\href{https://www.gesetze-im-internet.de/bgb/__376.html}{§ 376}} Rücknahme}}
		child { node {\textbf{\href{https://www.gesetze-im-internet.de/bgb/__378.html}{§ 378}} Ausschluss der Rücknahme}}
		}
	child { node {Leistung erüllungshalber}}
	child { node {\textbf{\href{https://www.gesetze-im-internet.de/bgb/__387.html}{§ 387}} Aufrechnung \\ Gegenseitigkeit \\ Gleichartigkeit \\ Fälligkeit \\ einredefreiheit}}
}
child[sibling angle=63] { node {Leistungs\-störungsrecht \\ \textbf{\href{https://www.gesetze-im-internet.de/bgb/__275.html}{§ 275}} Unmöglichkeit \\ Geld hat man zu haben! \\ \textbf{\href{https://www.gesetze-im-internet.de/bgb/__313.html}{§ 313}}}
		child[sibling angle=50] { node {Folgen}
			child { node {\textbf{\href{https://www.gesetze-im-internet.de/bgb/__326.html}{§ 326}} Befreiung von der Gegenleistung}}
			child { node {Anspruch auf Schadensersatz statt der Leistung}
				child { node {\textbf{\href{https://www.gesetze-im-internet.de/bgb/__311a.html}{§ 311a}} Leistungshindernis bei Vertragsschluss}}
				child { node {\textbf{\href{https://www.gesetze-im-internet.de/bgb/__280.html}{§ 280}} Schadensersatz wegen Pflichverletzung\\ \textbf{\href{https://www.gesetze-im-internet.de/bgb/__283.html}{§ 283}} Schadensersatz statt der Leistung bei Ausschluss der Leistungspflicht}}
				}
			child { node {\textbf{\href{https://www.gesetze-im-internet.de/bgb/__326.html}{§ 326}} Rücktritt vom Vertrag}}
			child { node {\textbf{\href{https://www.gesetze-im-internet.de/bgb/__285.html}{§ 285}} Herrausgabe des Ersatzes}}
			}
	child { node {Verzug}
		child { node {Verzug des Schuldners}}
		child { node {Verzug des Gläubigers}}
	}
	child { node {Schlechtleistung}}
	child { node {Verletzung einer Schutzpflicht}}
	}
child[sibling angle=63] { node {\textbf{\href{https://www.gesetze-im-internet.de/bgb/__320.html}{§ 320}} Einrede \\ Die zu erbringede Leistung kann so lange verweigert werden, bis der andere Teil seine Gegenleistun erbracht hat.}
	child { node {Vorraussetzungen}
		child { node {\href{\yt}{Gegenseitiger Vertrag}}}
		child { node {\href{\yt}{Gegensetigkeitsverhältnis (Synallagma)}}}
		child { node {\href{\yt}{Fälligkeit der Gegenforderung}}}
		child { node {\href{\yt}{Kein Ausschluss}}}
		}
		}
		;
\end{tikzpicture}



\href{https://www.youtube.com/watch?v=fu3dRQlV6og}{\cofeAm{0.3}{0.9}{159}{15cm}{7cm}}

\end{document}

